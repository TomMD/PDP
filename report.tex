\documentclass[11pt]{exam}

\usepackage{amsmath,inconsolata,verbatim}

\setlength{\parskip}{5pt}
\setlength{\parindent}{0pt}

\title{PDP/8 Simulator in Haskell}
\author{Thomas DuBuisson and Garrett Morris}
\date{February 13, 2012}

\def\Int{\texttt{Int12}}

\begin{document}
\maketitle

\section{Code Layout}
The Haskell programming language is strongly typed and allows
algebraic data types (ADTs, from here on called {\em types}).  We
created types representing the PDP/8 ISA, addressing modes, decoded
object files, 12 bit integers (\Int{}), and state of the PDP/8 machine.

To ease operations, the object files are first parsed into the object-file
format for loading into memory.  There is also an instruction decoding to
convert \Int{}s into the ISA type, used in the {\em fetch} operation.

Execution Unit - it executes.

All memory operation use the {\em load}, {\em store}, and {\em fetch} operations
provided by the Memory module.  By keeping these definitions in one place we can
easily check for proper memory logging and insert any desired debugging.

In its purest form, the Haskell language is void of mutable types.  To
simulate mutable memory of the machine state type, which contains
immutable registers as well as a mapping from addresses to values, we
use a ``State Monad'' which provides the illusion of mutable
operations while constructing and passing new, immutable, structures in
a manner transparent to the user of the monad.

\section{Test Strategy}
Early testing involved using a Haskell interpreter (GHCi) to
investigate internal data structures.  Once an operational PDP/8
interpreter was ready we could step through assembly instructions in
an interactive Read-Evaluate-Print Loop (REPL).  Entire programs,
listed below, were also tested and the results were compared to those
expected (i.e. known answer tests).  Property tests and random test
vectors were used to check small, but critical, sections of the code
(namely, the \Int{} type).

\subsection{Test Cases}
The test cases were mostly borrowed from the internet and a PDP/8
based book ``Learn to Program''.  Using the letters S, A, and I to
indicate the programs that test sub-routines, auto-increment, and
indirect addressing, our test programs included:

\begin{itemize}
\item add01.as - from the course web site.
\item countBinOnes.as - Tests basic control and rotation, from ``Learn to Program''.
\item euler24.as - A full program that computes the 63rd lexiographic ordering of
  the digits 0 through 9 (well-known problem euler-24). (SAI)
\item hello-world.as - A simple text-output test. (SAI)
\item printASC.as - A sub-routine driven ASC output test program.  (A)
\item jms.as - A trivial sub-routine test. (SI)
\item square.as - Compute the square of a two-digit octal number (uses IO).  (SI)
\end{itemize}

\section{Discovered Bugs}
\begin{itemize}
\item By stepping through instructions in the REPL
 \begin{itemize}
 \item Rotate left failed to set the link bit correctly.
 \item Direct addressing implemented as indirect addressing.
 \item Incorrect ASC translation on text entry.
 \end{itemize}

\item By interpreting the interpreter and observing our AST
 \begin{itemize}
 \item Invalid micro-op decoding
 \end{itemize}

\item By observing logs and statistics
 \begin{itemize}
 \item Invalid reporting of memory read operations
 \item Instruction totals $\neq$ total observed mnemonics
 \end{itemize}

\item By mixed methods
 \begin{itemize}
 \item ASM syntax bugs cause invalid object code.
 \item Incorrect cycle counting for some instructions.
 \end{itemize}

\item By property tests
 \begin{itemize}
 \item Absolute value of \Int{} types could result in out-of-bounds values.
 \end{itemize}
\end{itemize}

\end{document}
