\documentclass[11pt]{exam}

\usepackage{amsmath,verbatim,multicol}
\usepackage[compact]{titlesec}
\usepackage{mdwlist}

\setlength{\parskip}{5pt}
\setlength{\parindent}{0pt}
\setlength{\headsep}{0pt}
\setlength{\topskip}{0pt}
\setlength{\topmargin}{0pt}
\setlength{\topsep}{0pt}
\setlength{\partopsep}{0pt}

\title{PDP/8 Simulator in Haskell}
\author{Thomas DuBuisson and Garrett Morris}
% \date{February 13, 2012}

\def\Int{\texttt{Int12}}

\begin{document}
\maketitle

\section{Code Layout}
The Haskell programming language is strongly typed and allows
algebraic data types (ADTs, from here on called {\em types}).  We
created types representing the PDP/8 ISA, addressing modes, decoded
object files, 12 bit integers (\Int{}), and state of the PDP/8 machine.

To ease operations, the object files are first parsed into the object-file
format for loading into memory.  There is also an instruction decoding to
convert \Int{}s into the ISA type, used in the {\em fetch} operation.

Execution Unit - it executes.

All memory operation use the {\em load}, {\em store}, and {\em fetch} operations
provided by the Memory module.  By keeping these definitions in one place we can
easily check for proper memory logging and insert any desired debugging.

In its purest form, the Haskell language is void of mutable types.  To
simulate mutable memory of the machine state type, which contains
immutable registers as well as a mapping from addresses to values, we
use a ``State Monad'' which provides the illusion of mutable
operations while constructing and passing new, immutable, structures in
a manner transparent to the user of the monad.

\section{Features}
Features include:
{\small
\begin{itemize*}
\item Command line as well as REPL interfaces.
\item Flexible execution: single instruction stepping, multi-stepping, and execution that continutes until hitting a break-point.
\item Flexible setting and reading of memory and registers (including the switch register).
\item Logging of instruction counts, cycles, branch log, and memory accesses.
\end{itemize*}
}
\section{Test Strategy}
Early testing involved using a Haskell interpreter (GHCi) to
investigate internal data structures.  Once an operational PDP/8
interpreter was ready we could step through assembly instructions in
an interactive Read-Evaluate-Print Loop (REPL).  Entire programs,
listed below, were also tested and the results were compared to those
expected (i.e. known answer tests).  Property tests and random test
vectors were used to check small, but critical, sections of the code
(namely, the \Int{} type).

\subsection{Test Cases}
The test cases were mostly borrowed from the internet and a PDP/8
based book ``Learn to Program''.  Using the letters S, A, and I to
indicate the programs that test sub-routines, auto-increment, and
indirect addressing, our test programs included:

{\small
\begin{itemize*}
\item add01.as - from the course web site.
\item countBinOnes.as - Tests basic control and rotation, from ``Learn to Program''.
\item euler24.as - A full program that computes the 63rd lexiographic ordering of
  the digits 0 through 9 (well-known problem euler-24). (SAI)
\item hello-world.as - A simple text-output test. (SAI)
\item printASC.as - A sub-routine driven ASC output test program.  (A)
\item jms.as - A trivial sub-routine test. (SI)
\item square.as - Compute the square of a two-digit octal number (uses IO).  (SI)
\end{itemize*}
}

\section{Example Run}
An example run of the countbinOnes.as program:

\begin{multicols}{2}
{\scriptsize
\begin{verbatim}
# load test/countBinOnes.obj
# run
Program HALTed
# show stats
Total instructions: 19
Total cycles:       23
Breakdown:
                CLA 1
                CLL 3
                DCA 1
                HLT 1
                ISZ 2
                JMP 2
                RAL 3
                TAD 1
# show logs
Memory Log:
Instr Fetch 200
Instr Fetch 201
Data Write  215
Instr Fetch 202
Data Read   216
Instr Fetch 203
Instr Fetch 205
Instr Fetch 206
Instr Fetch 207
Instr Fetch 205
Instr Fetch 206
Instr Fetch 210
Instr Fetch 211
Data Read   215
Data Write  215
Instr Fetch 212
Instr Fetch 214
Instr Fetch 205
Instr Fetch 206
Instr Fetch 210
Instr Fetch 211
Data Read   215
Data Write  215
Instr Fetch 212
Instr Fetch 213

Branch Log:
203 205 SKP Taken
206 210 SKP Not taken
207 205 JMP Taken
206 210 SKP Taken
212 214 SKP Taken
214 205 JMP Taken
206 210 SKP Taken
212 214 SKP Not taken
213 215 SKP Not taken
\end{verbatim}
}
\end{multicols}

\end{document}
