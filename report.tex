\documentclass[11pt]{exam}

\usepackage{amsmath,verbatim,multicol}
\usepackage[compact]{titlesec}
\usepackage{mdwlist}
\usepackage[left=2.2cm,top=1cm,right=2.2cm,nohead,nofoot,bottom=1cm]{geometry}
\usepackage[compact]{titlesec}
\titlespacing{\section}{0pt}{*0}{*0}
\titlespacing{\subsection}{0pt}{*0}{*0}
\titlespacing{\subsubsection}{0pt}{*0}{*0}

\setlength{\parskip}{5pt}
\setlength{\parindent}{0pt}
%\setlength{\headsep}{0pt}
%\setlength{\topskip}{0pt}
%\setlength{\topmargin}{0pt}
%\setlength{\topsep}{0pt}
%\setlength{\partopsep}{0pt}

\title{PDP/8 Simulator in Haskell}
\author{Thomas DuBuisson and Garrett Morris}

\def\inl#1{\texttt{#1}}
\def\Int{\texttt{Int12}}

\begin{document}
\maketitle

\section{Code Layout}
The Haskell programming language is strongly typed and allows algebraic data types (ADTs, from here
on called {\em types}).  The \inl{Types} module contains definitions of types representing PDP/8
instructions (\inl{Instr}), addressing modes, decoded object files, 12 bit integers (\Int{}), and
tuples of values representing the state of the registers.  The \inl{Instr} type attempts to capture
the meaning, rather than the form, of PDP/8 instructions; for example, we use enumerations to
capture the various micro-operations, and attempt to exclude invalid combinations (such as \inl{RTR
  RTL}) in the type definition.  This significantly simplifies the \inl{Execute} module.  Similarly,
accessor functions, such the \inl{indirection} function that retrieves the indirection of an
operator, are only defined for memory instructions.  This means that improper use of them would
result in runtime errors rather than incorrect operation.

To ease operations, the object files are first parsed into the object-file format for loading into
memory.  There is also an instruction decoding to convert \Int{}s into the ISA type, used in the
{\em fetchInstruction} operation.

The \inl{Execute} module simulates the operation of the PDP/8 central processor.  We do not attempt
to simulate the CPMA (memory address) or MB (memory buffer) registers; rather, suitable values are
passed as arguments to the various functions in the \inl{Execute} module.  We do emulate some
details of the PDP/8 microarchitecture---for example, the \inl{PC} register is incremented
immediately after instruction fetch, and skips are then implemented with one further increment of
the \inl{PC} register.

Memory is modelled by a map from addresses to \Int{} values.  All memory operation are implemented
using the {\em load}, {\em store}, and {\em fetchInstruction} functions provided by the \inl{Memory}
module.  By keeping these definitions in one place we can easily check for proper memory logging and
insert any desired debugging.

Haskell does not directly support mutation.  We model the stateful operation of the PDP/8, such as
updating registers or memory, by using a {\em state monad}, which represents stateful computation in
a mutation-free fashion, translating intuitive operations such as \inl{getPC} or \inl{setPC} into
transition functions for an underlying state type.

\section{Features}
Features include:
\begin{itemize*}
\item Command line as well as REPL interfaces.
\item Flexible execution: single instruction stepping, multi-stepping, and execution that continutes until hitting a break-point.
\item Flexible setting and reading of memory and registers (including the switch register).
\item Logging of instruction counts, cycles, branch log, and memory accesses.
\end{itemize*}

\section{Test Strategy}
Early testing involved using a Haskell interpreter (GHCi) to
investigate internal data structures.  Once an operational PDP/8
interpreter was ready we could step through assembly instructions in
an interactive Read-Evaluate-Print Loop (REPL).  Entire programs,
listed below, were also tested and the results were compared to those
expected (i.e. known answer tests).  Property tests and random test
vectors were used to check small, but critical, sections of the code
(namely, the \Int{} type).

\subsection{Test Cases}
The test cases were mostly borrowed from the internet and a PDP/8
based book ``Learn to Program''.  Using the letters S, A, and I to
indicate the programs that test sub-routines, auto-increment, and
indirect addressing, our test programs included:

{\small
\begin{itemize*}
\item add01.as - from the course web site.
\item countBinOnes.as - Tests basic control and rotation, from ``Learn to Program''.
\item euler24.as - A full program that computes the 63rd lexiographic ordering of
  the digits 0 through 9 (well-known problem euler-24). (SAI)
\item hello-world.as - A simple text-output test. (SAI)
\item printASC.as - A sub-routine driven ASC output test program.  (A)
\item jms.as - A trivial sub-routine test. (SI)
\item square.as - Compute the square of a two-digit octal number (uses IO).  (SI)
\end{itemize*}
}

\section{Example Run}
An example run of the countbinOnes.as program:

\begin{multicols}{2}
{\small
\begin{verbatim}
# load test/countBinOnes.obj
# run
Program HALTed
# show stats
Total instructions: 19
Total cycles:       23
Breakdown:        
                CLA 1
                CLL 3
                DCA 1
                HLT 1
                ISZ 2
                JMP 2
                RAL 3
                SNA 3
                SNL 3
                TAD 1

# show logs
Memory Log:
Instr Fetch 200
Instr Fetch 201
Data Write  215
Instr Fetch 202
Data Read   216
Instr Fetch 203
Instr Fetch 205
Instr Fetch 206
Instr Fetch 207
Instr Fetch 205
Instr Fetch 206
Instr Fetch 210
Instr Fetch 211
Data Read   215
Data Write  215
Instr Fetch 212
Instr Fetch 214
Instr Fetch 205
Instr Fetch 206
Instr Fetch 210
Instr Fetch 211
Data Read   215
Data Write  215
Instr Fetch 212
Instr Fetch 213

Branch Log:
203 205 SKP Taken
206 210 SKP Not taken
207 205 JMP Taken
206 210 SKP Taken
212 214 SKP Taken
214 205 JMP Taken
206 210 SKP Taken
212 214 SKP Not taken
213 215 SKP Not taken
\end{verbatim}
}
\end{multicols}

\end{document}
